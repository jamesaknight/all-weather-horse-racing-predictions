\documentclass[twocolumn]{article}

% Packages
\usepackage[utf8]{inputenc}
\usepackage{amsmath, amssymb}
\usepackage{graphicx}
\usepackage{hyperref}
\usepackage{caption}
\usepackage{float}
\usepackage{geometry}

% Page Layout
\geometry{a4paper, margin=1in}

% Title and Author
\title{Predicting Horse Racing Odds with Neural Networks and Gradient Boosted Decision Trees: \\ A Multi-Method Approach to Modeling Sports Outcomes}
\author{Thomas Bale // James Knight}
\date{\today}

\begin{document}

% Title
\maketitle

% Abstract
\begin{abstract}
This paper explores the development of an all-weather horse racing odds predictor using neural networks (NNs) and gradient-boosted decision trees (GBDTs). Key innovations include the application of advanced optimization techniques, masking, attention mechanisms, and post-processing methods such as normal distribution modeling and softmax adjustments. With a consistent \(R^2\) score of 0.7, we discuss challenges and propose future directions, including comparative GBDT models for individual horses.
\end{abstract}

\section{Introduction}
\subsection{Background}
Sports betting, particularly in horse racing, presents unique challenges due to data sparsity and variability in racing conditions. Accurate odds prediction has both academic and commercial importance.

\subsection{Motivation}
The collaboration with a racing trading director highlights the practical need for robust predictive systems. By applying state-of-the-art machine learning methods, this paper aims to improve prediction accuracy and reliability.

\subsection{Contributions}
This work introduces:
\begin{itemize}
    \item Advanced optimization techniques (SGD, Adam, RMSProp, etc.).
    \item Masking and attention mechanisms for better input relevance.
    \item Post-processing methods like normal distribution modeling and softmax adjustments.
\end{itemize}

\section{Related Work}
Discuss the current literature on sports outcome prediction, including machine learning techniques in betting and horse racing. Highlight gaps this work addresses.

\section{Methodology}
\subsection{Dataset}
Describe the dataset, including features, pre-processing steps, and challenges like data sparsity. Explain how bootstrapping was applied to estimate variances.

\subsection{Model Architectures}
Detail the design of the neural network and GBDT models. Include specifics about:
\begin{itemize}
    \item Network architecture and hyperparameter choices.
    \item Optimization methods and their comparative performance.
\end{itemize}

\subsection{Post-Processing}
Explain techniques such as:
\begin{itemize}
    \item Modeling each horse's performance with a normal distribution.
    \item Using softmax adjustments for better comparative odds.
\end{itemize}

\subsection{Evaluation Metrics}
Discuss metrics like \(R^2\) and RMSE used for evaluating the models.

\section{Experiments and Results}
Present your experiments:
\begin{itemize}
    \item Comparison of baseline models vs. advanced methods.
    \item Impact of optimization techniques and post-processing.
    \item Evaluation of GBDT models for individual horses.
\end{itemize}
Include graphs and tables to visualize results.

\section{Discussion}
\subsection{Insights}
Analyze successes and failures of the models and techniques.

\subsection{Challenges}
Discuss obstacles such as overfitting, data sparsity, and model limitations.

\subsection{Future Work}
Propose directions for improvement, such as incorporating comparative data or exploring hybrid models.

\section{Conclusion}
Summarize the key contributions, findings, and implications for future research and practical applications in sports betting.

\section*{References}
\bibliographystyle{plain}
\bibliography{references}

\end{document}
